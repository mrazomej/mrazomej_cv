% TEX root = ../CV.tex
%-------------------------------------------------------------------------------
%	SECTION TITLE
%-------------------------------------------------------------------------------
\cvsection{Research Accomplishments}


%-------------------------------------------------------------------------------
%	CONTENT
%-------------------------------------------------------------------------------

\begin{cventries}

    %---------------------------------------------------------
      \cventry
        {with S. Barnes, N. Belliveau, T. Einav, M. Lewis, and R. Phillips} % collaborators
        {A Predictive Theory of Allosteric Induction} % Title
        {Cell Systems} % Journal
        {2018} % Date
        {
          \begin{cvitems} % Description(s) of tasks/responsibilities
            \item {
            \begin{flushleft}
            Despite lacking a nervous system, single bacterial cells are
            capable of making decisions given signals from their surroundings.
            How can individual molecules sense and transmit these signals? The
            answer comes from one of the crowning scientific achievements of the
            past century: allostery. Simply stated, allostery is the property of
            certain macromolecules to exist in multiple conformations with
            different properties. For example, transcription factors —proteins
            that control gene expression— can be active (able to bind the DNA)
            or inactive (unable to bind DNA) depending on the concentration of a
            signaling molecule.\linebreak
            In a 2018 paper, I wrote down a theoretical model that predicts the
            expression level of a gene regulated by an allosteric transcription
            factor. I then tested the model experimentally, showing that the
            model was able to predict how changes to the regulation of the gene
            translate to changes in the cellular response.\linebreak
            As we strive to turn synthetic biology into a predictive engineering
            discipline, this work brings us closer to be able to design
            input-output functions for biological circuits. The better models
            become, the easier it will be to design biological systems to
            address modern society problems.
          \end{flushleft}
            }
          \end{cvitems}
        }
        \cventry
        {with S. Marzen, G. Chure, R. Taubman, M. Morrison, and R. Phillips} % collaborators
        {First-principles prediction of the information processing capacity of a simple genetic circuit} % Title
        {Physical Review E} % Journal
        {2020} % Date
        {
          \begin{cvitems} % Description(s) of tasks/responsibilities
            \item {
            \begin{flushleft}
              Living organisms are constantly sensing intra and extracellular
              cues and responding accordingly. The quality and precision of such
              responses can mean the difference between surviving or not certain
              challenges; therefore, there is a constant selection pressure for
              cells to gather enough information from any stimulus in order to
              build an adequate response. In this context, the information that
              cells can obtain has a precise mathematical definition measured
              —just as in computers— in bits.\linebreak
              In a recent publication our goal was to predict how many bits of
              information can a cell harboring a simple genetic circuit process.
              To do so, I wrote down a theoretical model to predict the full
              distribution of gene expression based on the physics of this
              molecular process. I calibrated our model with previous
              information in order to perform parameter-free predictions. To
              test the model, I compared the predictions with experimental
              single-cell gene expression data finding great agreement.\linebreak
              This work puts us a step closer towards understanding the
              intriguing idea for a general principle in biology: information,
              in its mathematical definition, is a quantity that cells try to
              maximize over evolutionary time.\linebreak
          \end{flushleft}
            }
          \end{cvitems}
        }
    \end{cventries}