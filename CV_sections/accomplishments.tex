% TEX root = ../CV.tex
%-------------------------------------------------------------------------------
%	SECTION TITLE
%-------------------------------------------------------------------------------
\cvsection{Research Accomplishments}


%-------------------------------------------------------------------------------
%	CONTENT
%-------------------------------------------------------------------------------

\begin{cventries}

    %---------------------------------------------------------
      \cventry
        {with S. Barnes, N. Belliveau, T. Einav, G. Chure, M. Lewis, and R. Phillips} % collaborators
        {A Predictive Theory of Allosteric Induction} % Title
        {Cell Systems} % Journal
        {2018} % Date
        {
          \begin{cvitems} % Description(s) of tasks/responsibilities
            \item {
            \begin{flushleft}
            We wrote down a theoretical model, based on statistical physics,
            that predicts the expression level of a gene regulated by an
            allosteric transcription factor. We then tested the model
            experimentally by measuring the mean gene expression of a series of
            strains with different biophysical parameters, showing that the
            model was able to predict how changes to the regulation of the gene
            translate to changes in the cellular response.
          \end{flushleft}
            }
          \end{cvitems}
        }
      \cventry
        {with G. Chure, N. Belliveau, T. Einav, Z. Kaczmarek, S. Barnes, M. Lewis, and R. Phillips} % collaborators
        {Predictive shifts in free energy couple mutations to their phenotypic
        consequences} % Title
        {PNAS} % Journal
        {2019} % Date
        {
          \begin{cvitems} % Description(s) of tasks/responsibilities
            \item {
            \begin{flushleft}
            We expanded our model of gene regulation by allosteric transcription
            factors to explore changes in biophysical parameters due to changes
            in the amino-acid sequence of the transcription factor. We mutated
            both the DNA-binding domain and the inducer-binding domain of the
            protein, and re-fit the corresponding parameters given new gene
            expression measurements. We then predicted the double-mutants by
            adding the corresponding free-energy changes for each of the
            mutations, and found great agreement with the corresponding
            experimental data.
          \end{flushleft}
            }
          \end{cvitems}
        }
        \cventry
        {with S. Marzen, G. Chure, R. Taubman, M. Morrison, and R. Phillips} % collaborators
        {First-principles prediction of the information processing capacity of a simple genetic circuit} % Title
        {Physical Review E} % Journal
        {2020} % Date
        {
          \begin{cvitems} % Description(s) of tasks/responsibilities
            \item {
            \begin{flushleft}
              With the objective of predicting the amount of information (in
              bits) that a simple genetic circuit can gather from the state of
              the environment, we wrote down a non-equilibrium model for the
              full distribution of gene expression as a function of an external
              signal. We calibrated our model with previous information in order
              to perform parameter-free predictions. To test the model, we
              compared the predictions with experimental single-cell gene
              expression data finding great agreement.
          \end{flushleft}
            }
          \end{cvitems}
        }
      \cventry
        {with M. Morrison and R. Phillips} % collaborators
        {Reconciling Kinetic and Equilibrium Models of Bacterial Transcription} % Title
        {PLoS Comp. Bio. (accepted)} % Journal
        {2020} % Date
        {
          \begin{cvitems} % Description(s) of tasks/responsibilities
            \item {
            \begin{flushleft}
              We did an exhaustive analysis of the correspondence between
              equilibrium models of gene regulation --based on statistical
              mechanics-- and equivalent kinetic models --based on the chemical
              master equation. We found that both frameworks make
              indistinguishable predictions at the level of mean gene
              expression. Only through accounting for higher moments of the
              distribution we can get mechanistic insights into which scheme
              better describes the regulation of the gene. We finally used
              Bayesian inference to show that "the best model" we found can
              predict single-molecule mRNA counts distributions.
          \end{flushleft}
            }
          \end{cvitems}
        }
      \cventry
        {$\;$} % collaborators
        {Research interests} % Title
        {$\;$} % Journal
        {$\;$} % Date
        {
          \begin{cvitems} % Description(s) of tasks/responsibilities
            \item {
            \begin{flushleft}
              I am interested in transitioning into evolutionary biology.
              Specifically, I find the parallels between population genetics and
              statistical physics extremely intriguing. I am therefore
              interested in using my background on theoretical modeling,
              Bayesian statistical inference, and molecular biology to explore
              questions on microbial evolution from a theory-experiment dialogue
              angle.  
          \end{flushleft}
            }
          \end{cvitems}
        }
\end{cventries}