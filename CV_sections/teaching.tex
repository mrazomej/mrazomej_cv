% TEX root = ../CV.tex
%-------------------------------------------------------------------------------
%	SECTION TITLE
%-------------------------------------------------------------------------------
\cvsection{Teaching}


%-------------------------------------------------------------------------------
%	CONTENT
%-------------------------------------------------------------------------------
\begin{cventries}

%---------------------------------------------------------
  \cventry
    {Teaching assistant} % Role
    {BE262: Physical Biology Bootcamp} % Event
    {Caltech} % Location
    {2017, 2028, 2019 \& 2020 (1 week each)} % Date(s)
    {
      \begin{cvitems} % Description(s)
        \item {Description: This course provides an intensive introduction to 
        thinking like a quantitative biologist. Students will use Python to 
        write computer code to analyze real-world quantitative experiments. No
        previous experience in coding is presumed, though for those with 
        previous coding experience, advanced projects will be available. 
        Students will use “street fighting mathematics” to perform order of 
        magnitude estimates on problems ranging from how many photons it takes 
        to make a cyanobacterium to the forces that can be applied by 
        cytoskeletal filaments. These modeling efforts will be complemented by 
        the development of physical models of phenomena such as gene expression,
        phase separation in nuclei and cytoskeletal polymerization.}
      \end{cvitems}
    }

  \cventry
    {Teaching assistant} % Role
    {Bi/Ge105: Evolution} % Event
    {Caltech} % Location
    {2018 \& 2020 (10 weeks)} % Date(s)
    {
      \begin{cvitems} % Description(s)
        \item {Description: The objective of this course is to use broad brush
        strokes to paint a picture of modern thinking on evolution, both of
        living organisms and the planet they inhabit. The first part of the
        course takes a decidedly historical and naturalistic perspective,
        focusing on the timeline of evolution (based on the geological record
        of fossils and geochemical signatures) and how the natural history of
        both the inanimate and animate worlds in the hands of Darwin and
        Wallace (and their distinguished predecessors) led to the articulation
        of the tenets of evolution by natural selection as well as a picture of
        a dynamic Earth. This is followed by an examination of the emergence of
        modern genetics which gave us a molecular picture of variation. Next,
        we undertake a study of the forces of evolution such as selection,
        drift, migration and mutation and how population genetics provides a
        quantitative framework for examining the interplay between these
        forces. With these preliminaries in hand, we will then turn to a
        variety of case studies in evolution that illustrate the principles of
        the subject with some of the remarkable studies that have been made to
        test these ideas.}
      \end{cvitems}
    }
  \cventry
    {Teaching assistant} % Role
    {Physical Biology of the Cell @ Bangalore} % Event
    {NCBI, India} % Location
    {2019 (1 week)} % Date(s)
    {
      \begin{cvitems} % Description(s)
        \item {Description: This intensive week long mini-course explores the
        way that physical and mathematical models can be used to understand
        biological systems. The course begins by examining the way in which
        simple order of magnitude estimates can be used to provide insights
        into problems ranging from the fidelity of protein translation to how
        far a bird can fly without stopping to how amphibians arrive on oceanic
        islands. This is followed by the use of statistical mechanics to
        explore problems in regulatory biology. Some examples include the
        physics of post-translational modifications, how cells make
        transcriptional decisions and the precision with which embryonic
        development takes place. In addition to these topics, the course also
        involves a series of hands-on projects using Python to amplify the case
        studies discussed in class.\\}
      \end{cvitems}
    }

  \cventry
    {Teaching assistant} % Role
    {Physiology Course} % Event
    {MBL, Woods Hole} % Location
    {2016, 2017 \& 2018 (3 weeks each)} % Date(s)
    {
      \begin{cvitems} % Description(s)
        \item {Description: The MBL Physiology Course is one of the oldest
        continually running biology courses in the world. The Course
        traditionally has had three goals that we strongly endorse: graduate
        training, cutting edge research, and introducing new generations of
        scientists to the unique environment of the MBL. The modern vision is a
        Course that brings together biological and physical / computational
        scientists, both in the faculty and the student body, to work together
        on cutting edge problems in cell physiology. These interactions create
        an environment that is more of a summer school in interdisciplinary
        science than a conventional graduate course.}
      \end{cvitems}
    }

  \cventry
    {Teaching assistant} % Role
    {Physical Biology of the Cell} % Event
    {MBL, Woods Hole} % Location
    {2016, 2017 \& 2018 (3 weeks each)} % Date(s)
    {
      \begin{cvitems} % Description(s)
        \item {Description: The course explores the description of a broad
        array of topics from modern biology using the language of physics and
        mathematics. It focuses on physical and mathematical model building by
        drawing examples from broad swaths of modern biology including cell
        biology (signaling and regulation, cell motility), physiology
        (metabolism, swimming and flight), developmental biology (patterning of
        body plans, how size and number of organelles and tissues are
        controlled), neuroscience (action potentials and ion channel gating,
        vision) and evolution (population genetics, biogeography).}
      \end{cvitems}
    }

  \cventry
    {Teaching assistant} % Role
    {BE/Bi103: Data Analysis in the Biological Sciences} % Event
    {Caltech} % Location
    {2015 \& 2017 (10 weeks each)} % Date(s)
    {
      \begin{cvitems} % Description(s)
        \item {Description: Modern biology is a quantitative science, and
        biologists need to be equipped with tools to analyze quantitative data.
        This course takes a hands-on approach to developing these tools. Among
        other techniques, we show how to do regression, parameter estimation,
        outlier detection and correction, error estimation, hypothesis testing,
        denoising, and image processing and quantification}
      \end{cvitems}
    }

  \cventry
    {Teaching assistant} % Role
    {BE/Bi101: Order of Magnitude Biology} % Event
    {Caltech} % Location
    {2015 (10 weeks)} % Date(s)
    {
      \begin{cvitems} % Description(s)
        \item {Description: Students develop skills in the art of educated
        guesswork and apply them to the biological sciences. Building from a
        few key numbers in biology, students will ?size up? biological systems
        by making inferences and generating hypotheses about phenomena such as
        the rates and energy budgets of key biological processes. The course
        will cover the breadth of biological scales: molecular, cellular,
        organismal, communal, and planetary.}
      \end{cvitems}
    }

  \cventry
    {Teaching assistant} % Role
    {Evolution @ GIST} % Event
    {GIST, South Korea} % Location
    {2015 \& 2018 (1 week each)} % Date(s)
    {
      \begin{cvitems} % Description(s)
        \item {Description: Students develop skills in the art of educated
        guesswork and apply them to the biological sciences. Building from a
        few key numbers in biology, students will ?size up? biological systems
        by making inferences and generating hypotheses about phenomena such as
        the rates and energy budgets of key biological processes. The course
        will cover the breadth of biological scales: molecular, cellular,
        organismal, communal, and planetary.}
      \end{cvitems}
    }

  \cventry
    {Instructor} % Role
    {``Biolog\'{i}a a trav\'{e}s de los n\'{u}meros" (biology by the numbers)} 
    {Clubes de Ciencia Mexico} % Location
    {2014 (1 week)} % Date(s)
    {
      \begin{cvitems} % Description(s)
        \item {Intense biophysics bootcamp to introduce senior high-school
        students, 1st \& 2nd year undergraduate students to the current
        challenges in quantitative biology.}
      \end{cvitems}
    }

  \cventry
    {Teaching assistant} % Role
    {Bi1X: The great ideas of biology, an introduction through experimentation}
    {Caltech} % Location
    {2014 (10 weeks)} % Date(s)
    {
      \begin{cvitems} % Description(s)
        \item {Description: Bi1x provides students with an introduction to
        concepts and laboratory methods in biology. Molecular biology
        techniques and advanced microscopy were combined to explore the great
        ideas of biology.}
      \end{cvitems}
    }

%---------------------------------------------------------
\end{cventries}
