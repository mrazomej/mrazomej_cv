%!TEX TS-program = xelatex
\documentclass[]{friggeri-cv}

\begin{document}
\header{\hspace{25 mm}Manuel Razo-Mejia}{}
       {\hspace{40 mm}Graduate Student - Biochemistry \& Molecular Biophysics. Caltech}


% In the aside, each new line forces a line break
\begin{aside}
	\section{contact}
		1200 E. California Blvd.\\
		91125 MC 114-96\\
		Pasadena, CA.
  		mrazomej at caltech dot edu\\
  		(626) 590 3634
	
	\section{interests}
	physical biology\\
	systems biology\\
	computational biology\\
	evolution
	
	\section{languages}
		spanish-native\\
		english-fluent\\ 
		TOEFL iBT 116/120
    
    	\section{programming skills}
		fluent in Python\\
    		R (bioconductor)\\matlab\\mathematica\\shell scripting
\end{aside}


\section{education}
	\textbf{PhD Biochemistry \& Molecular Biophysics (2014-present)}\\
	California Institute of Technology, Pasadena, CA\\
    	\textbf{BSc Biotechnological Engineering (2009-2014)}\\
    	Instituto Politecnico Nacional, Silao, Guanajuato, Mexico.\\
	{\small\addfontfeature{Color=lightgray}Related coursework: Bioengineering, thermodynamics, molecular biology, separation processes, general physics, differential equations, vector calculus, statistics \& probability.}}\\

\section{research experience}
{\small\addfontfeature{Color=lightgray}June - August 2015}}\\
\textbf{Physiology course, MBL, Woods Hole, MA.}\\

{\small\addfontfeature{Color=lightgray}June - August 2013}}\\
\textbf{Weizmann Institute of Science, Rehovot, Israel.}\\
Advisor: \textbf{Ron Milo}, Assistant Professor.\\
Metabolic engineering and synthetic biology to increase the natural growth rate of the bacteria \textit{E. coli} on the highly oxidized glyoxylate molecule as part of a more general effort of designing an autotrophic organism from a heterotrophic backbone.\\
{\small\addfontfeature{Color=lightgray}General skills acquired: Gene library assembling, automated high-throughput phenotypic screening, principles of metabolic engineering.}\\

{\small\addfontfeature{Color=lightgray}June 2012 - June 2013}}\\
\textbf{California Institute of Technology, Pasadena, CA.}\\
Advisor: \textbf{Rob Phillips}, Fred and Nancy Morris Professor of Biophysics and Biology.\\
Implementation of a thermodynamic model of gene regulation in the wild type lac operon to explore the variability in the regulation of this genetic circuit; mapping changes in the DNA bases into changes in the parameters of the model and predicting the corresponding phenotypic changes. By analyzing \textit{E. coli} natural isolates from all over the world we explored the natural variability of this genetic circuit among different isolates from the same species.\\
{\small\addfontfeature{Color=lightgray}General skills acquired: Biophysical modeling based on Statistical Mechanics, \textit{E. coli} molecular biology (gene expression level measurement, sequencing, quantification of proteins via immuno-dots).}\\

{\small\addfontfeature{Color=lightgray}2011/2013}\\
\textbf{National Laboratory of Genomics for Biodiversity, Irapuato, Mexico.}\\
Advisor: \textbf{Cei Abreu-Goodger}, Assistant Professor.\\
 Computational analysis of the conservation of microRNAs targets between zebrafish and mouse using bioinformatics and statistical  methods.\\
{\small\addfontfeature{Color=lightgray}General skills acquired: shell scripting, unix based operating system management, bioinformatic tools such as bioconductor, programing skills.}\\

{\small\addfontfeature{Color=lightgray}December 2011 - January 2012}\\
\textbf{Department of Systems Biology, Harvard Medical School, Boston, MA.}\\
Advisor: \textbf{Michael Springer}, Assistant Professor.\\
Development of a method based on PCR amplifications to identify 64 different fully sequenced natural isolated \textit{S. cerevisiae} strains, based on their mutation profiles.\\
{\small\addfontfeature{Color=lightgray}General skills acquired: Yeast molecular biology (transformation, genomic DNA extraction, etc.).}\\

\section{publications}
\textbf{Razo-Mejia, M.}, Boedicker, J. Q., Jones, D., DeLuna, a, Kinney, J. B., & Phillips, R. (2014). \textit{Comparison of the theoretical and real-world evolutionary potential of a genetic circuit}. Physical Biology, 11(2), 026005. doi:10.1088/1478-3975/11/2/026005\\

Lior Zelcbuch, \textbf{Manuel Razo-Mejia}, Elad Herz, Sagit Yahav, Niv Antonovsky, Hagar Kroytoro, Ron Milo, Arren Bar-Even. (submitted). \textit{An in vivo metabolic approach for deciphering the product specificity of glycerate kinase proves that both E. coli's glycerate kinases generate 2-phosphoglycerate}.


\section{awards/scholarships}
Amgen Graduate Fellowship\\
{\addfontfeature{Color=lightgray}Caltech-Amgen Research Collaboration, 2015}\\
\\
Benjamin M. Rosen Graduate Fellowship\\
{\addfontfeature{Color=lightgray}California Institute of Technology, 2014}\\
\\
Valedictorian, Class of 2014\\
{\addfontfeature{Color=lightgray}Instituto Politecnico Nacional, 2014}\\
\\
Summer Kupcinet-Getz International Science School Fellowship\\
{\addfontfeature{Color=lightgray}Weizmann Institute of Science, 2013}\\
\\
Summer Undergraduate Research Fellowship\\
{\addfontfeature{Color=lightgray}California Institute of Technology, 2012}\\

\section{teaching experience}
Instructor - ``Biologia a traves de los numeros" (biology by the numbers).\\
Description: Intense biophysics bootcamp to introduce senior high-school students, 1st \& 2nd year undergraduate students to the current challenges in quantitative biology.\\
Duration: 1 week, 10-12 hours per day.\\
{\addfontfeature{Color=lightgray}Clubes de Ciencia Mexico. Ensenada Baja California, 2014}\\

Teacher assistant - Bi1X: The great ideas of biology, an introduction through experimentation.\\
Description: Bi1x provides students with an introduction to concepts and laboratory methods in biology. Molecular biology techniques and advanced microscopy were combined to explore the great ideas of biology.\\
Duration: 10 weeks, 6 hours per week.\\
{\addfontfeature{Color=lightgray}California Institute of Technology, 2013}\\

\section{extracurricular activities}
\textbf{Cubes de Ciencia Mexico. Organization committee}\\
The emerging non-profit association Clubes de Ciencia Mexico aims to expand the access of young Mexican students to high quality scientific education. For this we design and implement science, technology, engineering and math workshops for high school students and freshman undergrads. Our science clubs are the mechanism to establish a network of mentors that link the most prominent young scientists in Mexico and abroad with other Mexican students interested in science. This international network of mentors tries to catalyze three main objectives:\\
(1) Increase the interest for science among the students.\\
(2) Guide young students towards scientific careers.\\
(3) Develop science-related technical and cognitive abilities.

\end{document}
